\documentclass{article}

\usepackage[utf8]{inputenc}
\usepackage[T1]{fontenc}
\usepackage{geometry}
\usepackage{lscape} %hoja horizontal
\geometry{a4paper}
\usepackage[frenchb]{babel}

%\title{Premier document}
%\author{Un TeXnicien}
%\date{}

\begin{document}


%\maketitle
\begin{titlepage}
	\begin{center}
		
		\line(1,0){200}\\
		\huge{\bfseries Reporte Anual}\\
		\line(1,0){200}\\


	\end{center}
\end{titlepage}

Voici un texte accentué en françaisdfgdfg!\\
%[]
[10 mm] % se puede poner en cm, mm y in
Voici un texte accentué en françaisdfgdfg!\\

\begin{flushright} % texto del lado derecho

	Este texto está del lado derecho

\end{flushright}

\section{Tamaño de papel a utilizar en documentclass}
 Ancho 	Alto \\
 letterpaper 210mm 297mm\\
legalpaper 148mm 210mm\\
executivepaper 176mm 250mm\\
a4paper 8.5in 11in\\
a5paper 8.5in 14in\\
b5paper 7.25in 10.5in\\

\section{Orientación del papel en documentclass}
portrait: vertical
landscape: horizontal

\begin{landscape}
	Texto en hoja horizontal
\end{landscape}

\section{Impresión}
A excepción de la clase book, los demás documentos se imprimen a una sola cara, pero este se puede modificar, se pone en documentclass
oneside: para una cara
twoside: para dos caras

creción de cpaitulos o secciones
\section{Capítulos o secciones}
no hay necesidad de poner tamaño, letra o num de capitulo

el estilo Fancy tiene seis campos, ters en el encabezado y tres en el pié de página.
\lhead{}: campo de la izquierda del encabezado
\chead{}: campo central del encabezado. Quedará centrado
\rhead{}: campo de la derecha del encabezado. Quedará alineado a la derecha 

\lfoot{}: campo de la izquierda del pie
\cfoot{}: campo central del pie. Quedará centrado
\rfoot{}: campo de la derecha del pie. Quedará alineado a la derecha 

Agregando líneas
El paquete fancyhdr también permite insertar líneas horizontales para separar estos elementos del resto de la página. Podemos definir el grosor de estas líneas mediante los comandos:
\renewcommand{\headrulewidth}{0.45pt}
\renewcommand{\footrulewidth}{1pt}

En el encabezado
\usepackage{fancyhdr} %activamos el paquete
\pagestyle{fancy} %seleccionando un estilo
\pagestyle{headings} %agregar el nombre de la seccion y el número en el encabezado

\foot{autor}
\renewcommand{\headrulewidth}{0.4pt} % grosor de la línea de la cabecera
\renewcommand{\footrulewidth}{0.4pt} % grosor de la línea del pie

Tarea
¿Cómo poner una imagen en cabecera y pie de página?

Subtemas y subsubtemas
\subsection{subseccion}
\subsubsection{subsubseccion}

Parrafos y sub párrafos
Por debajo del nivel subsection tambien es posble agregar subparrafos y parrafos. estas unidades aparecen sin número y se declaran mediante los comandos:
\paragraph{}
\subparagraph{}

Dividir un libro en partes
En caso de querer dividir un documento en partes es necesario utilizar las clases 'book' o 'report'
Estas divisiones pueden introducirse mediante los comandos \part{} y \chapter{}

\part{criptografia}
	\chapter{cripto clasica}
	\section{cripto de sustitucion}
		\subsection{cifrado monoalfabetico}
		\subsection{cifrado polialfabetico}
	\section{cripto de transportacion}
		\subsection{escitalo espartano}

\part{Esteganografía}

	\chapter(este clásica)
	\section{tinta visible}
	\section{reduccion fotográfica}
	\chapter{esteganografia moderna}
	\section{marcas de agua visibles}
	\section{marcas de agua ocultas}


\section{Cómo hacer una tabla?}
\begin{table}

F\centering
\label{tab: DESrobustez}
\caption{Robustez de algoritmo DES}
\begin{tabular}{l c r}
Algoritmo & tipo de cifrado & robustez \\hline
Vigenere & simétrico & baja \\
RSA & simétrico & alta \\
\end{tabular}

\end{table}

\section{Tabal en lugar de cuadro}
Cuando trbajamaos con el paquete spanish, las tablas son nombradas 'cuadros'; para cambiar el nombre a tablas:\\
\usepackage[spanish, es-tabla]{babel}

\section{Indice de tablas}
Elindice de tablas pude crearse con el cmando 0lisoftables. Al igual que en el caso del {indice de figuras es necesario haber definido el t{itulo de las tablas que queremos que aparezca en el índice mediante el comando \caption

%indice de tablas
\listoftables
\addcontentsline{toc}{section}{\numberline}{lista de tablas}
\clearpage

\section{Tarea}
¿Cómo modificar el ancho de las columnas?

\sectionmovernos desde el índice con un clic}
Con hidelinks podemos movernos desde el índice a donde se encuentra el capítulo o figura, tabla...
\usepackage[hidelinks]{hyperref} 

\section{texto con color}
cuando quremos poner de un color determinado cierta palabra, podemos usar el paquete {xcolor}, que incorpora funciones para definir el colo del texto con \textcolor, usamos el paquete:\\
\usepackage[usenames, dvipsnames, svgnames]{xcolor}

usenames: permite llanar a los 16 colores básicos por su nombre
dvipsname da acceso a otros 64 colores extra
svgnames, añade 150 colores más

Como se dijo en el capitulo \ref{secuno}
\textcolor{BlueGreen}{texto con color}
\lipsum
Tambien podemos definir nuestros propios colores. Para ello se deben usar los sig protocolos:\\
RGB = mezcla rojo verde y azul, cada uno entre 0 y 255
rgb=  mezcla de cuan, magenta y amarullo y negro. cada uno con unvalor entre 0 y 1
gray= escala de grises: entre 0 (negro)  y 1 (BLANCO)

\end{document}